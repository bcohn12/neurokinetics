\section{DISCUSSION}
\subsection{Strategy optimization in the frequency domain}
We know that when given a repeated task, there is sometimes a stable frequency that some people think of as optimal. How plastic is optimality
How quickly does a neuromechnaical system adjust to changing situations that might affect the cost landscape?
In Selinger et. al. 2015's work, they grasped a way of quantifying how a walking human changes her/his step frequency when biomechanical constraints are changed (i.e. when a higher step frequency is made artificially more energetically costly).
With this, we see a family of solutions which solve the problem, with varying metabolic cost (Net metabolic power)

\subsection{what is a motor program?}
Some people refer to a motor program as a set of control 'modes' which change.
Movement science and force-generation are different tasks, so some might say they have different motor programs.

\subsection{how does the timeframe of motor control plasticity get affected by subspaces?}
If for a given task our feasible control strategy is not highly constrained (many choices), moving across that space may take a long time.
If suddenly half of the space is cut off, and the individual's preferred motor program used a solution in the missing family of solutions, it will be a long haul to jump back inside the solution space.

Our results provide evidence supporting the following:
\begin{itemize}
	\item{Firstthing}
	\item{Secondthing}
	\item{Thirdthing}
\end{itemize}
