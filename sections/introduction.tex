
\section{INTRODUCTION}

For a musculoskeletal system with $n$ muscles and $m$ output dimensions, the set of all feasible activations for a given static task \textbf{b} is represented here as
$A\textbf{x} \leq b$, where $A \in \mathbb{R}^{m \times n}$ represents the set of linear constraints upon the system, $x \in \mathbb{R}^n$ represents the set of activations capable of meeting those requirements.


We applied our approach to these things:

1. A schematic model
2. A realistic model from [briantodo cite]

With this, below are the key observations we identified with our research:
\begin{itemize}
\item{item}
\item{item}
\item{item}
\end{itemize}

With respect to the structure of the activation space, we set forth the following key ideas and findings:
\begin{itemize}
\item{item}
\item{item}
\item{item}
\end{itemize}
and most importantly,
\begin{itemize}
\item{item}
\end{itemize}
