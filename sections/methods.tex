\section{METHODS}
[briantodo put in the models that we applied our things to]
% subsection minimization_of_change_in_activation (end)
\subsection{Minimization of total activation}
\label{sub:minimization_of_total_activation} % (fold)
We are given $k$ discrete time steps and its corresponding output constraints and activation constrains
\begin{align*}
A^1 x^1 &= b^1 \\
A^2 x^2 &= b^2 \\
 &\vdots \\
A^k x^k &= b^k \\
x^j &\in [0,1]^n \text{ for all } j \in \{1,2,\dots,k\} 
\end{align*}
Additionally we are given spatio temporal constraints in the form of $|x_i^{j+1} - x_i^{j}| \leq \delta_i^j|$ for $i \in \{1,2,\dots,n\}, j \in \{1,2,\dots,k-1\}$, i.e.\ the activation of muscle $i$ can not change more than $\delta_i^j$ between step $j$ and step $j+1$. Note that $|x_i^{j+1} - x_i^{j}| \leq \delta_i^j$ is not a linear constraint itself but can easily be replaced by the two constraints $x_i^{j+1} - x_i^{j} \leq \delta_i^j$ and $x_i^{j} - x_i^{j+1} \leq \delta_i^j$

Finding a solution where we minimize total activation is hence given by

\begin{equation}
\begin{array}{lrcl}
\mbox{minimize} & \sum_{j=1}^k \sum_{i=1}^n x_i^j \\ 
\mbox{subject to} & A^j x^j &=& b^j \hspace{5mm} \forall j \in \{1,\dots, k\} \\
  & x_i^{j+1} - x_i^{j} &\leq& \delta_i^j  \hspace{5mm} \forall j \in \{1,\dots, k-1\}, i \in \{1,\dots,n\} \\
  &  -(x_i^{j+1} - x_i^{j}) &\leq& \delta_i^j  \hspace{5mm} \forall j \in \{1,\dots, k-1\}, i \in \{1,\dots,n\} \\
  & x^j &\in& [0,1]^n \hspace{5mm} \forall j \in \{1,\dots, k\}.
\end{array}
\end{equation}

This is a linear program in $kn$ variables.



% subsection minimization_of_total_activation (end)

\subsection{Minimization of change in activation} % (fold)
\label{sub:minimization_of_change_in_activation}
The activation change of muscle $i$ between time step $j$ and $j+1$ is given by
\[|x_i^{j+1} - x_i^{j}|.\]
We are now interested in the solution of the above system, where the sum of the change of activations is minimized. I.e.\ we minimize $\sum_{j=1}^{k-1}\sum_{i=1}^n |x_i^{j+1} - x_i^{j}|$, under the output constraints, activation constraints and spatio-temproal constraints. The linear program therefore looks as follows:

\begin{equation}
\begin{array}{lrcl}
\mbox{minimize} & \sum_{j=1}^{k-1} \sum_{i=1}^n |x_i^{j+1} - x_i^{j}| \\ 
\mbox{subject to} & A^j x^j &=& b^j \hspace{5mm} \forall j \in \{1,\dots, k\} \\
  & x_i^{j+1} - x_i^{j} &\leq& \delta_i^j  \hspace{5mm} \forall j \in \{1,\dots, k-1\}, i \in \{1,\dots,n\} \\
  &  -(x_i^{j+1} - x_i^{j}) &\leq& \delta_i^j  \hspace{5mm} \forall j \in \{1,\dots, k-1\}, i \in \{1,\dots,n\} \\
  & x^j &\in& [0,1]^n \hspace{5mm} \forall j \in \{1,\dots, k\}.
\end{array}
\end{equation}

This is technically not a linear program since the objective function consists of absolute values of linear terms. By a standard trick this problem can be captured in a linear program. Instead of minimizing $\sum_{j=1}^{k-1} \sum_{i=1}^n |x_i^{j+1} - x_i^{j}|$ we minimize $\sum_{j=1}^{k-1} \sum_{i=1}^n e_i^j$ and add constraints 
\begin{align*}
e_i^j &\geq x_i^{j+1} - x_i^{j} \\
e_i^j &\geq - (x_i^{j+1} - x_i^{j}).
\end{align*}

For each term $|x_i^{j+1} - x_i^{j}|$, we add an auxiliary variable $e_i^j$. The constraints guarantee that 
\[e_i^j \geq \max\{x_i^{j+1} - x_i^{j}, - (x_i^{j+1} - x_i^{j})\} = |x_i^{j+1} - x_i^{j}|.\]
For an optimal solution each of these inequalities has to be satisfied with equality, otherwise we could decrease the corresponding $e_i^j$. The linear program on $kn + (k-1)n$ variables that we are interested in solving is hence given by

\begin{equation}
\begin{array}{lrcl}
\mbox{minimize} & \sum_{j=1}^{k-1} \sum_{i=1}^n e_i^j \\ 
\mbox{subject to} & A^j x^j &=& b^j \hspace{5mm} \forall j \in \{1,\dots, k\} \\
  & x_i^{j+1} - x_i^{j} &\leq& \delta_i^j  \hspace{5mm} \forall j \in \{1,\dots, k-1\}, i \in \{1,\dots,n\} \\
  &  -(x_i^{j+1} - x_i^{j}) &\leq& \delta_i^j  \hspace{5mm} \forall j \in \{1,\dots, k-1\}, i \in \{1,\dots,n\} \\
  & x^j &\in& [0,1]^n \hspace{5mm} \forall j \in \{1,\dots, k\} \\
  & e_i^j &\geq& x_i^{j+1} - x_i^{j} \hspace{5mm} \forall j \in \{1,\dots, k-1\}, i \in \{1,\dots,n\}\\
  & e_i^j &\geq& -(x_i^{j+1} - x_i^{j}) \hspace{5mm} \forall j \in \{1,\dots, k-1\}, i \in \{1,\dots,n\}.
\end{array}
\end{equation}

For muscle $i$ at moment $j$, $\delta_i^j$ is equal to the amount of change in normalized activation ($a \in [0,1]$) possible up to the \textbf{next} time point; we chose to define \delta as future capability of change. Because of this, there will be $k-1$ total values of delta used; \delta at the last timepoint will not be used in our computations.